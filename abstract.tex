\chapter*{\textit{Abstract}}

The uptake of cryptocurrencies has soared in recent years; growing rapidly in popularity and mainstream adoption. At the dismay of law enforcement, adopters of this new technology include criminals wishing to evade the regulatory oversight of traditional payment mechanisms in order to participate in illegal trade (including, but not limited to, drugs, hacks, ransomware and even murder-for-hire \cite{RefWorks:doc:5cfbc8fee4b05850fa02e710}). The largest cryptocurrency of all, Bitcoin, is estimated to be involved in \$76 billion of illegal activity each year \cite{RefWorks:doc:5cfbc8fee4b05850fa02e710}. Those using Bitcoin for illegal means aren't trivially identifiable; Bitcoin identities are pseudo-anonymous, such that they are not tied to any real-world entity, but all transactions they are involved in are publicly visible and entirely transparent. 
\\\\
In this project, we develop a tool to assist with conducting digital forensic investigations across Bitcoin. Although there do exist some commercial tools which have the same goal, they are proprietary and not available to the wider community. To develop our tool, we leverage the transparency of the Bitcoin's Blockchain to build a graph database which stores, and represents the relationships between, Bitcoin's 420+ million transactions. The graph database is the foundation to building \textit{Radar}, a web-based digital forensic investigation tool. \textit{Radar} presents historical Bitcoin transactions through interactive, graph-based visualisations. With the support of intuitive graph-navigation and filtering controls, \textit{Radar} can be used to efficiently navigate historical Bitcoin activity. We further associate addresses with the entities which control them using both existing datasets and idiom of use clustering heuristics. Working with investigators at the Metropolitan Police and Coinbase, we evaluate the \textit{Radar} potential as a commercial digital forensic tool. 