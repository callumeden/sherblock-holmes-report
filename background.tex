\chapter{Background}

\section{Bitcoin}
The model of traditional banking is based on a centralised trusted authority \cite{RefWorks:doc:5c39e80ae4b0854ae611b047}; we use a third party that we trust in order to mediate the transfer of funds, a process in which we must explicitly define the payee and the payer (i.e. we must identify our friend and ourselves, if we were transferring money to a friend). Bitcoin has no central authority, it is a distributed peer-to-peer system. Therefore, sending money to our friend as before no longer requires any mediation; we send it directly to them. However, transferring funds peer-to-peer of course possesses several challenges; how do you prevent double spending? How do we ensure money is not counterfeit? \cite{RefWorks:doc:5c39e80ae4b0854ae611b047} How is currency issued without a central authority? 
\\\\
Bitcoin solves these challenges by relying on the \textit{Blockchain} [see \ref{background-blockchain}] for its 'source of truth', which itself is secured through \textit{mining} [see \ref{background-mining}] and network-wide consensus \cite{RefWorks:doc:5c39e80ae4b0854ae611b047}. All \textit{transactions} [see \ref{background-transactions}] are publicly available to view on the Blockchain and can be verified by any \textit{node} [see \ref{background-nodes}]. 

\subsection{Addresses, Keys \& Hashing}
A fundamental concept to Bitcoin is public key cryptography. The basic idea is that we generate a private and public key pair. We can pick a private key at random and use that to generate a public key (using \gls{elliptic-curve-crypto}). The public key can then be used to receive funds [see transactions \ref{background-transactions}] and the private key used to sign transactions to spend the funds. \cite{RefWorks:doc:5c39e80ae4b0854ae611b047}. It is critical to Bitcoin security that the process of generating a public key from a private key is one way and using a public key, it should be impossible to generate the private key. 
\\\\
A Bitcoin address can be generated from a public key. This Bitcoin address is the only information that a user needs to offer in order to receive payments, and since the address essentially appears as a random combination of letters and characters, it does not \textit{directly} identify the user which generated the address. However, since all transactions are published publicly, Bitcoin only has pseudo-anonymity [see \ref{background-anonymity}].

\subsection{Bitcoin Address Types}\label{background-address-types}
A Bitcoin address may begin with a 1, a 3 or a bc1 which distinguishes between different types of addresses. A Bitcoin address beginning with 1 indicates a Pay-to-PubKey-Hash address. An address beginning with 3 indicates a Pay-to-Script-Hash address. Addresses beginning with bc1 are a Bech32 type address (a segregated-witness address) \cite{RefWorks:doc:5c39e80ae4b0854ae611b047}.

\subsection{Vanity Addresses}\label{background-vanity-addresses}
Vanity addresses typically begin with characters that contain human readable messages; for example, the address \texttt{1LoveBPzzD72PUXLzCkYAtGFYmK5vYNR33} is a vanity address as it begins with the message 'love' \cite{RefWorks:doc:5c39e80ae4b0854ae611b047}. Vanity addresses are generated by generating and testing billions of candidate private keys, generating the corresponding public key and comparing the public key to the desired pattern until a match is found. Vanity addresses are typically used to create a more distinctive address, for example for businesses to reassure customers they are paying the correct address and to attempt to prevent adversaries substituting their own address in order to steal payments. 

\subsection{Blockchain} \label{background-blockchain}
The Blockchain is the global ledger for Bitcoin. It contains the entire Bitcoin history, from the \gls{genesis} to the most recently mined block. It exists as an ordered, back-linked list of blocks of transactions \cite{RefWorks:doc:5c39e80ae4b0854ae611b047} where each block is linked to its previous block, known as its parent block, up to the genesis block. 
\\\\
A block contains a header, containing metadata, then a list of transactions that are included in that block \cite{RefWorks:doc:5c39e80ae4b0854ae611b047} (see figure \ref{fig:simplified-block-structure}). A block can be uniquely identified by a cryptographic hash of its header. The metadata of the block contains data such as the previous block hash (used to create the chain link to the parent described earlier), a difficulty target and the nonce (used in the process of mining explained in \ref{background-proof-of-work}).  

\begin{figure}[h!]
  \centering
  \includegraphics[width = 15cm]{./figures/block-structure-simple.pdf}\\[0.5cm]
  \caption{A simplified structure of a Bitcoin block. \cite{RefWorks:doc:5c39e80ae4b0854ae611b047}}
  \label{fig:simplified-block-structure}
\end{figure}

\subsection{Mining}\label{background-mining}
Mining is the mechanism that underpins the decentralised clearinghouse of transactions and also the mechanism in which new bitcoin is (currently) issued. Miners, incentivised by the reward of bitcoin, continuously compete to solve the \textit{proof-of-work algorithm} [see \ref{background-proof-of-work}] for each block. 
\\\\
As soon as a miner finds a solution for a block, they propagate this to the network, allowing each node to independently verify the block. In order to receive compensation for their efforts, a miner must include in the block a special transaction from the \textit{coinbase} [see \ref{background-coinbase}] to their own public address. If the block is valid, then the block will be added to the blockchain and eventually the miner will be remunerated with newly minted bitcoin by the special \textit{coinbase} [see \ref{background-coinbase}] transaction. The successful miner also claims the transaction fees for the transactions included in the mined block. For all nodes that see this newly mined block, and is part of their longest chain, the next round of mining begins immediately, again competing to find the solution of the next blocks proof-of-work algorithm. 

\subsection{Coinbase}\label{background-coinbase}
The coinbase refers to the input of a special type of transaction (the coinbase transaction) which, unlike regular transactions, does not consume bitcoin (i.e. it has no 'spending element'). Rather, it has a singular input called the \textit{coinbase}. The bitcoin is therefore created out of nothing. This is the mechanism in which new bitcoin are introduced into the network. 

\subsection{Proof of work}\label{background-proof-of-work}
New blocks are mined approximately every 10 minutes; this rate of mining is maintained even with fluctuations in the hash-rate of the network by periodically adjusting the difficulty of the proof-of-work algorithm. A change in hash-rate can be attributed to an increased/reduced amount of computational power of the network; this can be caused by a change in the number of participants or advancements in mining hardware.
\\\\
As shown in figure \ref{fig:all-time-hash-rate} the hash-rate of Bitcoin has grown exponentially in recent years, reflecting an increase in the number of nodes contributing to the network in addition to advances in hardware. There is, however, a noticeable decline in hash-rate quite recently (around the time of November 2018) due to a \textit{hard-fork} [see \ref{background-forks}]. Comparing the hash-rate to difficulty over time (see figure \ref{fig:all-time-difficulty}) demonstrates how Bitcoin's difficulty is dynamically adjusted in response to changes in the networks hash-rate. 
\\\\
So, what exactly is the proof-of-work algorithm that miners are trying to solve? Firstly, observe in figure \ref{fig:simplified-block-structure} there exists a 'nonce' field. This exists as a variable that can be changed freely by a miner. As simply put as possible, the miner wants to find a hash for the block that is lower than some target value. This can be achieved by repeatedly adjusting the nonce field and generating the new resulting block hash and comparing it to the target hash. Since a hash function is one-way, we have the property that a miner can only find a hash lower than the target is by performing repeated trial and error of nonces/hashes (i.e iterating through different values for the nonce and looking at the hash of the block with that nonce value).
\\\\
Clearly, the problem of finding a hash lower than a target can have its difficulty adjusted by adjusting how small or large the target is. A lower target will have fewer hashes that are smaller than it, and therefore a lower probability of finding a satisfying hash; the difficulty and the target are inversely related. 
\\\\
In summary, the Proof-of-Work problem creates a measure of computational effort that must have been spent in order to achieve a valid solution. Although it is theoretically possible to find a satisfying solution immediately, it is incredibly unlikely. By adjusting the difficulty accordingly, Bitcoin achieves an average of ~10 minutes' worth of the networks computational effort being spent for each solution that is found. 

\begin{figure}[h!]
  \centering
  \includegraphics[width = 15cm]{./figures/hashrate-all-time}\\[0.5cm] 
  \caption{Bitcoin hash-rate for all of time \protect\footnotemark. }
  \label{fig:all-time-hash-rate}
\end{figure}
\footnotetext{https://www.blockchain.com/charts/hash-rate?timespan=all}
\begin{figure}[h!]
  \centering
  \includegraphics[width = 15cm]{./figures/difficulty-all-time}\\[0.5cm] 
  \caption{Bitcoin difficulty for all of time \protect\footnotemark.}
  \label{fig:all-time-difficulty}
\end{figure}
\footnotetext{https://www.blockchain.com/charts/difficulty}

\subsection{Transactions}\label{background-transactions}
Transactions are the core component facilitating the transfer of funds in Bitcoin. A transaction tells the network that the owner of some bitcoin has authorised the transfer of that bitcoin denomination to another owner \cite{RefWorks:doc:5c39e80ae4b0854ae611b047}. In the simplest case, transactions contain inputs, which represent where the funds are coming from (the sender), and outputs, which represent where the funds are going to (the new owner). 
\\\\
For the transfer of funds to be confirmed, the transaction must be added to the global ledger for everyone to see. This means it must be included in a block that is mined on the Blockchain. To be included in a future block, the transaction must be propagated to the many nodes of the network. The creator of the transaction must, therefore, send it to some of the Bitcoin nodes it knows the location of (its neighbours). These nodes, if the transaction is valid, will propagate the transaction to their neighbours, which will repeat the same process. This is the process of 'flooding' the network with the transaction. In addition to sending the bitcoin value to the recipient, the sender must also include an incentive for the miner to perform the work to include the transaction in the block they are currently mining. This is the \textit{transaction fee}.
\\\\
The transaction fee is a small value that is implicitly included in the transaction by leaving some left-over funds from the inputs once the recipient of the transaction has been 'paid'. This fee is collected by the miner who includes the transaction in the block they mined. A larger fee will act as a higher incentive for a miner to include the transaction in the block and will likely result in the transaction being included in the Blockchain in a timelier fashion.  

\subsection{Nodes}\label{background-nodes}
There are several different types of nodes in Bitcoin. One type of node is the 'full node'. Full nodes maintain a copy of the entire Blockchain locally, containing all transactions. These nodes must maintain and build their copy of the Blockchain by listening to incoming transactions and blocks. A full node can independently verify each transaction and block using its own copy of the Blockchain, without relying on information from other nodes. The disadvantage of running a full node is that it consumes a large amount of storage. 
\\\\
Not all nodes require a full copy of the Blockchain, and in many cases cannot hold a full copy due to resource constraints on devices such as smart-phones. These nodes are called Simplified Payment Verification (SPV)\cite{RefWorks:doc:5c39e80ae4b0854ae611b047} nodes. These nodes do not have a full picture of the history of Bitcoin (i.e. all of the transactions) but do know all of the hashes of the blocks in the Blockchain. An SPV node, therefore, uses the depth of the block that the transaction is in to verify it, in comparison to the full node which will build a full database of unspent bitcoin from the transaction to the genesis block and verify it is not a double spend. The SPV node will wait until the transaction is in a block at depth of at least 6, relying on the knowledge that other nodes have accepted the transaction as confirmation that the transaction is valid. 

\subsection{Immutable History}
The Blockchain ledger becomes more and more immutable as time passes; this is because, in order to add a different block into the Blockchain, you must expend effort to re-do the proof-of-work for that block. Then, in order to change a block buried under other blocks, you must also expend the energy to re-do the proof-of-work for all of the blocks it is buried under (since you now require a different parent block hash field value and will then get a different hash for child blocks). This makes it exponentially less likely to be able to catch up with the main chain as the chain grows \cite{RefWorks:doc:5c3b547fe4b0613d0cda0434}. Therefore, the more blocks a block is 'buried' by, the more immutable it becomes, making old blocks in the chain are practically immutable \cite{RefWorks:doc:5c39e80ae4b0854ae611b047}. 

\subsection{Mining Pools}
As touched upon earlier in mining [see \ref{background-mining}], a miner must expend enormous amounts of computational effort in order to compete for the proof-of-work solution of a block, resulting in them being compensated for their efforts with new bitcoin. 
\\\\
A miner without a substantial amount of resources at their disposal (i.e. their individual contribution to the network's hash-rate is near-negligible) will be unlikely to successfully mine a block and be compensated for their efforts. It may be the case that they must mine for many years before they are successful. This makes investing in hardware and energy a large gamble for many prospective miners; they will likely prefer smaller, but more frequent enumeration for their efforts. Mining pools are the solution to this problem for many miners.
\\\\
Mining pools essentially combine many miners' resources in order to mine a block. Using many participants, a pool will have a much higher likelihood of being successful in mining a block. The reward of bitcoin is then paid to the pool, which the pool then distributes amongst the contributing members of the pool. A pool measures contributions of miners by giving the miners a much lower target for the block they are mining than the actual, more difficult, Bitcoin target. This will allow miners to mine blocks with the goal of finding a solution for the lower pool target, which the pool can recognise and use this as a measure of their contribution. Occasionally a pool miner will mine a block that meets the lower pool target and also meets the network's target; the pool will then propagate this solution to the network, claim the reward and distribute it according to miners contributions (taking a cut for itself, of course). 

\subsection{Forks}\label{background-forks}
A fork on the Blockchain can occur naturally; an inherent property of a decentralised network, such as Bitcoin, is that different nodes can have different views of the world (i.e. due to transmission delays). However, these forks are usually quickly corrected within a small number of (usually one) blocks \cite{RefWorks:doc:5c39e80ae4b0854ae611b047}. There also exist forks that have been deliberately created, i.e. by an attacker attempting to re-write the history of the Blockchain, or by a \textit{hard-fork} software release. 
\\\\
A hard-fork in the Blockchain is where the Bitcoin network permanently diverges. A hard-fork occurs when there is a change in the consensus rules. The consensus rules tell a node which blocks to accept, and which ones to reject. A block must conform to the consensus rules of the majority of nodes in order to be added to the chain. When a change to consensus rules occurs, not all nodes may be on-board. Some may still be running with the old consensus rules. This would mean they would create and accept blocks using the old rules rather than the new ones - meaning their blocks will not be accepted by nodes running with the new rules (hard-forks mean a non-backwards compatible change). This leads to a divergence in the chain, where one chain is based on blocks added by nodes running the old rules, and the other chain containing blocks added by nodes running with the new rules. 
\\\\
In a similar vein, there also exist 'soft forks' which do not lead to a divergence of the Blockchain since they incorporate a backwards-compatible change. This means that blocks added by nodes running the old software will not be rejected by those running the new software. 
\\\\
Figure \ref{fig:hard-fork} shows how a hard-fork will look on the Blockchain; the fork at block 3 represents a naturally occurring fork, as mentioned at the beginning of this section, whereas the fork occurring at block 6 is a hard-fork and may be due to a change in the consensus rules of the network \cite{RefWorks:doc:5c39e80ae4b0854ae611b047}. 

\begin{figure}[h!]
  \centering
  \includegraphics[width = 15cm]{./figures/hard-fork.pdf}\\[0.5cm] 
  \caption{An example of a hard-fork \cite{RefWorks:doc:5c39e80ae4b0854ae611b047}.}
  \label{fig:hard-fork}
\end{figure}


\section{Anonymity}\label{background-anonymity}

Since all transactions to occur in Bitcoin are available to view by the public, Bitcoin can only offer pseudo-anonymity rather than real anonymity \cite{RefWorks:doc:5c3db7d6e4b0fa2b1fe68b48}. There exist studies which show it is possible to de-anonymise Bitcoin transactions based on data that is publicly available \cite{RefWorks:doc:5c3db7d6e4b0fa2b1fe68b48}. Therefore, there exist services called 'mixers' which aim to obfuscate transaction origins with the goal of strengthening privacy and make such de-anonymisation more challenging. 

\subsection{Mixing Services}\label{background-mixing-service}
A mixing service, also known as 'mixer' or 'tumbler', works in the following way: A user wishing to use a mixing service will first create a new address and send bitcoin to an address of the service, asking the service to send the funds back to their new address. Other users also using the mixing service will take the same steps and send bitcoin to the service. The service now holds bitcoin for multiple users. The service can now use any of the addresses in which it holds bitcoin to send money back to the users of the service. This results in the appearance of a disconnect between the user's old address (the one which held the bitcoin initially) and the new address which now holds their bitcoin \cite{RefWorks:doc:5c3dace5e4b0613d0cda512b}. Clearly, this helps disguise the origins of bitcoin and can be used as a tool in the process of money laundering. The operators of these services profit by charging a fee in exchange for 'mixing' their bitcoin. 

\subsection{Risks of using transaction anonymisers}
In order for a mixing service to provide functionality, it will likely retain a history of the senders and recipients. If an attacker wishing to discover users using these services were to set up a Mixer, they could potentially gain full knowledge of relationships between senders and recipients if logs were to be kept for this data over a large enough time span \cite{RefWorks:doc:5c3dace5e4b0613d0cda512b}. Of course, this would only be effective if the user relies on a single mixing service; a user could mitigate this vulnerability by using multiple mixing services, but at the cost of paying more fees. 
\\\\
Other weaknesses in anonymisation services could be the timing of incoming and outgoing transactions, in addition to transaction values, which could all be used to correlate the senders and recipients of bitcoin. Furthermore, the communication between the user and the service itself, if compromised, could reveal information (such as addresses) used to de-anonymise transactions. 
\\\\
In addition, studies have shown some other mixing services such as Bitlaunder, DarkLaunder and Coinmixer to have multiple serious security flaws and can be easily exploited to compromise the privacy of those that use it. In fact, the investigation shows that making a mixer that is genuinely secure is a difficult task, which may be refreshing news for law enforcement wishing to taint bitcoin back to its source, but worry-some for legitimate users of such mixing services \cite{RefWorks:doc:5c3db214e4b0854ae6124c26}. 


\subsection{Peeling Chain}\label{background-peeling-chain}
A peeling chain is a pattern of use that exists widely in the Bitcoin network; it can be used in withdrawals from exchange services and mining pools, and in some cases, it forms part of a signature of illegal activity. The chain begins at an address that often holds a large bitcoin value, and the goal is to obfuscate the funds that a wallet holds. This is achieved by a series of transactions in which the bitcoin will be sent to two or more addresses, one of which will belong to the service (which owns the originating address) where the majority of the bitcoin will be sent, and the small remainder to some change addresses. 
\\\\
Recently, peel chains are being used less frequently due to modern Blockchain analysis software developing the capability to collapse even the most complex peel chains. There also exists transaction fees at each step of the peel chain, which may make them economically inefficient. 
\begin{figure}[h!]
  \centering
  \includegraphics[width = 10cm]{./figures/peel-chain}\\[0.5cm] 
  \caption{An example of a peel chain. \cite{RefWorks:doc:5c3db214e4b0854ae6124c26}}
  \label{fig:peel-chain}
\end{figure}


\subsection{Taint Analysis}\label{background-taint}
Taint analysis is a measurement of the link a denomination of cryptocurrency has with previous illegal activity. For instance, if a vendor has accepted a payment of bitcoin where one of the 3 inputs to the transaction was stolen in a theft, some part of the output of that transaction will have a 'taint' measurement to show its link to the theft, even though the vendor in this example knows nothing of the theft. Taint analysis, therefore, impacts the fungibility of bitcoin. 
\\\\
Taint analysis was offered in a feature for free by blockchain.info. The definition of 'taint analysis' on the site was \lq Taint is the \% of funds received by an address that can be traced back to another address\rq. The taint therefore usually correlates with the percentage of funds that are linked to some theft of coins or are known to have been used in some illicit manner. However, this feature is no longer available on the site, with some speculation to the feature being retracted in order to provide it as a charged, premium offering.

\subsection{TOR}
The onion router (TOR) is network infrastructure used to obscure geographical locations of IP addresses by routing communications through multiple proxies located around the globe. Messages are sent in multiple layers of encryption for each relay, so each relay can decrypt the outer layer and forward the decrypted resulting message onto the next router. Eventually, the one-layer encrypted message will reach its destination. Using Tor therefore permits for protection against eavesdropping and traffic analysis \cite{RefWorks:doc:5c3e103be4b014f3944e4192}.

\section{Bitcoin address clustering (on-chain)}\label{background:clustering}
It is possible to define a number of heuristics that can be used in an effort to group distinct Bitcoin addresses so that they can be collectively associated with an individual user. This section outlines some of these heuristics that use data from the Blockchain to help cluster addresses belonging to the same user.

\subsection{Multi-input transactions}\label{background:multi-input-tx}
Multi-input transactions will be required when some user, say Alice, wishes to transfer some funds, say the value of $v$ to some other user Bob; however, Alice does not hold a single bitcoin denomination that is greater or equal to $v$ and therefore must combine multiple smaller denominations to meet/exceed the value of $v$. 
\\\\
It is, therefore, a safe assumption to say the owners of each input of a transaction belong to the same user, regardless of the distinct addresses each input is locked to \cite{RefWorks:doc:5c3de14be4b042abd3bcc2c6}. This heuristic, therefore, leverages this assumption to cluster together addresses whenever they are the inputs of the same transaction.

\subsection{Change Addresses}\label{background:cluster-change-address}
A transaction will often generate change when the value of the inputs exceeds the amount to be paid to the recipient and any transaction fees, and the remainder will want to be paid back to the sender. To do this, Bitcoin will generate a change address for the remainder to be sent to. Therefore, it can usually be safely assumed that when a transaction has two outputs, and one is a new address that has not appeared before, that this address is the change address and belongs to the sender of bitcoin for this transaction \cite{RefWorks:doc:5c3de14be4b042abd3bcc2c6}. 
\\\\
A weakness in using this heuristic as highlighted in previous research \cite{Refworks:doc:5c3de7e3e4b0ea6196452d80} is that it is not robust in the face of changing patterns of use in the network since it is an idiom of use rather than an inherent property of Bitcoin. In fact, a study by Sarah Meiklejohn et al \cite{Refworks:doc:5c3de7e3e4b0ea6196452d80} found that falsely linking just a small number of change addresses causes entire relationship graphs to collapse into giant clusters that are not actually controlled by a single user. 
\\\\
However, through careful investigation of false positives and implementing a more conservative clustering algorithm, it was possible to name 1,600 times more addresses post-clustering than those already identified through manual tagging \cite{Refworks:doc:5c3de7e3e4b0ea6196452d80}. Evidently, these heuristics could be extremely useful in the process of identifying users and tracking funds, such as by collapsing peeling chains [see peeling chain \ref{background-peeling-chain}].

\subsection{Consumer Wallet Heuristic}\label{background:consumer-wallet-heuristic}
This heuristic as presented in 'Data-Driven De-Anonymization in Bitcoin' by Jonas David Nick \cite{RefWorks:doc:5cfa2acee4b0132e0223d6f7} uses the assumption that popular consumer wallets (such as  Bitcoin Core, Electrum, MultiBit, Armory, Android Bitcoin Wallet, etc) by default only allows users to send bitcoins to a single address. Therefore, transactions generated by a consumer wallet will only have one or two outputs; one to the recipient address and one as the change address (if change was required). 
\\\\
This heuristic can then be used in clustering by identifying change addresses: for every public address $p$, find the transactions $ts$ in which an output locked to $p$ is spent (i.e $p$ spends bitcoin) then ensure every $t$ in $ts$ has less than 3 outputs. If $p$ is spent by a transaction that has more than 2 outputs, it is not a change address \cite{RefWorks:doc:5cfa2acee4b0132e0223d6f7} . 

\subsection{Optimal Change Heuristic}\label{background:optimal-change-heuristic}
This heuristic is also presented in 'Data-Driven De-Anonymization in Bitcoin' by Jonas David Nick \cite{RefWorks:doc:5cfa2acee4b0132e0223d6f7}. It relies on the assumption that wallet software does not spend unnecessary outputs when constructing transactions, since including more outputs will necessary will lead to unnecessary bloat in the size of the transaction and therefore higher transaction fees.
\\\\
Using this assumption, you can assume that the change value will be smaller than any of the input values; since if this wasn't true and it was larger, the smaller input could just be omitted from the transaction and the change output value will be reduced by this amount. Therefore, if a transaction has a unique output which is smaller than all inputs, it is very likely to be the true 'optimal change output' \cite{RefWorks:doc:5cfa2acee4b0132e0223d6f7}.


\subsection{Behaviour based analysis}\label{background:behaviour-clustering}
Humans naturally fall exhibit behaviour patterns, and since many events on the Blockchain are human-driven, it is possible to attempt to identify these behaviour patterns and apply them in clustering the activities of distinct users. Using timestamps and network properties, it becomes possible to observe such behaviour patterns could include items being purchased, daily schedule and activities, in addition to non-human behaviour patterns of hardware and network latencies \cite{RefWorks:doc:5c3f3459e4b042abd3bceede}. 
\\\\
The study 'Identifying Bitcoin users by transaction behaviour' \cite{RefWorks:doc:5c3f3459e4b042abd3bceede} shows that a transaction can be described by its timestamp, connectivity (number of inputs/outputs) and coin flow. Features can be extracted that help characterise some aspect of transaction behaviour over time. These features are: 
\begin{itemize}
    \item The time interval between successive transactions
    \item The hour of the day the transaction took place
    \item The time of hour (seconds since the start of the hour)
    \item The time of day (seconds since the start of the day)
    \item Coin flow - the net bitcoin value  
    \item Input/output balance - The balance of inputs from other users compared to outputs going to other users.
\end{itemize}

\section{Bitcoin address clustering (off-chain)}
The previous section outlines heuristics for using on-chain data for address clustering, however, it is possible that there exists more public information available elsewhere on the internet that can be used to help cluster addresses \cite{RefWorks:doc:5c3ef27be4b03c7dd82ce4e6}. 

\subsection{Tag Collection}\label{background-tag-collection}
Tag collection is the process of trying to find a Bitcoin address that is mentioned in the same data frame as some tag (such as a username or a company name) \cite{RefWorks:doc:5c3ef27be4b03c7dd82ce4e6}. Therefore, passive tag collection can be carried out by crawling sites where this information is likely to appear, such as social media, bitcoin forums and dark-web marketplaces (for example, Silkroad). Tagging in this manner may allow addresses to be correlated to known Bitcoin businesses, or categorised in the type of service they are, such as an exchange, marketplace, mining pool, mixer, gambling etc. 

\subsection{Entity Clustering}
Sarah Meiklejohn et al's work \cite{RefWorks:doc:5c3de7e3e4b0ea6196452d80} is an example of how addresses can be clustered to be under the control of known entity by proactively engaging with the services through transactions, which require a public address of the entity. This provides the minimal 'ground truth' data needed to bootstrap the formation of larger clusters using the on-chain heuristics in section \ref{background:clustering}.

\section{Popular Services}
\subsection{Satoshi Dice}
A very popular Bitcoin dice game, introduced in April 2012, is Satoshi Dice. Users may place bets and, if they win, have some multiple of their bets paid back to them \cite{Refworks:doc:5c3de7e3e4b0ea6196452d80}. It is estimated that Satoshi Dice is responsible for about 60\% of activity on the Bitcoin network and is expected to contribute an extra 14MB to the overall Blockchain daily \cite{Refworks:doc:5c3de7e3e4b0ea6196452d80}. 

\subsection{Exchanges}
Often it is unavoidable to use an exchange service. Exchange services enable a user to exchange their currency, such as a \gls{fiat-money} or another cryptocurrency, into bitcoin and vice versa. For those users attempting to user bitcoin for illicit means, this poses an issue as it is a point of centralisation; an example is a user who has stolen bitcoin, who wishes to then convert stolen funds to their fiat currency, but first must go through a known exchange. It's possible their bitcoin is now tainted [see \ref{background-taint}] and will not be accepted by many exchanges. 

\section{Illegal Activity}

The impact Bitcoin can have in facilitating illegal activity is substantial; in a 2012 intelligence report \cite{RefWorks:doc:5c4ad055e4b0ea619646c15a}, the FBI claimed: 

\begin{displayquote}
 ``If Bitcoin stabilizes and grows in popularity, it will become an increasingly useful tool for various illegal activities beyond the cyber realm. For instance, child pornography and Internet gambling are illegal activities already taking place on the Internet which require simple payment transfers. Bitcoin might logically attract money launderers, human traffickers, terrorists, and other criminals who avoid traditional financial systems by using the Internet to conduct global monetary transfers.''
\end{displayquote}

The report now 7 years ago feels almost like an early warning call; since then we have seen the FBI's predictions materialise as we see high-profile thefts, money laundering and drug-buying stories hit headlines. 
\\\\
Why does Bitcoin make life difficult for law enforcement? Bitcoins decentralised nature introduces a number of vulnerabilities to traditional law enforcement techniques. There is a lack of anti-money-laundering software, account owners often cannot be directly identified and it is more difficult to identify the original source of funds \cite{RefWorks:doc:5c4ad055e4b0ea619646c15a}. 

\subsection{Silk Road}
Silk road was an international online marketplace which operates as a Tor hidden service that was overwhelmingly used as a market for controlled substances and narcotics \cite{RefWorks:doc:5c3e0105e4b0854ae612621e}. Through investigation by Nicolas Christin \cite{RefWorks:doc:5c3e0105e4b0854ae612621e}, it appeared to be quite an advanced marketplace including features for a product rating system with customer feedback, an escrow account for dispute management, automatic price pegging to the USD for sellers (accounting for bitcoin price fluctuations) and site-wide promotional campaigns such as 'pot day' promoting the sale of cannabis on the site \cite{RefWorks:doc:5c3e0105e4b0854ae612621e}. However, in 2013 the FBI shut down Silk Road, but an almost identical site Silk Road 2.0 quickly emerged and began generating sales of \$8 million a month with 150,000 active users. Just a year later, the FBI also shut down Silk Road 2.0 and arrested it's operator \cite{RefWorks:doc:5c4ae400e4b0613d0cdbb201}. 

\todo{Money Laundering}
Bitcoin can be used for the purpose of money laundering; funds routed through \textit{mixers} [see \ref{background-mixing-service}] can conceal the origin and destination of funds such that it can be traced by law enforcement. This could ultimately facilitate perpetrators in the act of concealing or mischaracterizing the proceeds of crime (a.k.a money laundering). 

\subsection{Bitcoin ATM's}
Bitcoin ATM's are physical machines, often installed in small shops, that allow users to buy and sell various cryptocurrencies by trading in their \gls{fiat-money}. For example, there exist many Bitcoin ATM's in London, UK where you buy bitcoin in exchange for depositing pound sterling. The owners of these machines have the incentive of hosting them in their stores as they can charge a commission fee on these transactions. Over the last few years, these ATM's have become increasingly abundant across the UK, as shown in figure \ref{fig:background-atm-growth}. 
\\\\
However, there have been growing concerns that these ATM's are being used in the process of money-laundering. In a December 2018 report by Bloomberg \cite{RefWorks:doc:5c4af4bfe4b0686b56fa4839}, reporters carry out an experimental investigation into the anti-money laundering measures being taken by Bitcoin ATM's; estimating that more than half of the machines in the US are not following the rules. Many machines do not verify identification or impose limits on transactions, in direct violation of several US banking laws \cite{RefWorks:doc:5c4af4bfe4b0686b56fa4839}. 

\begin{figure}[h!]
  \centering
  \includegraphics[width = 10cm]{./figures/atm-installs-uk}\\[0.5cm] 
  \caption{Accumulated number of crypto ATMs installed over time - UK only \protect \footnotemark}
  \label{fig:background-atm-growth}
\end{figure}
\footnotetext{https://coinatmradar.com/charts/growth/united-kingdom/}

\subsection{Significant Thefts}\label{background:significant-thefts}
Not only is Bitcoin used as a mechanism to facilitate off-chain illegal activity; crime also occurs on-chain in the form of Bitcoin thefts. Thefts are quite common in Bitcoin, and those of significant value often make the headlines. Below are some major thefts \footnote{https://bitcointalk.org/index.php?topic=576337}:

\begin{center}
\begin{tabular}{ |p{4cm}|p{4.5cm}|p{2cm}|p{3cm}|} 
 \hline
\textbf{Theft Name} & \textbf{Theft Victims} & \textbf{Approx Loss (\bitcoinA)} & \textbf{Theft date} \\\hline
Mass MyBitcoin Thefts & MyBitcoin users with weak account passwords & 4019 & June 20th-21st 2011 \\\hline

MyBitcoin Theft & MyBitcoin \& customers & 78739 & July 2011 \\\hline

Bitcash.cz Hack &  Bitcash.cz & 484 & November 11th, 2013 \\\hline

Linode Hacks & Bitcoinica, Bitcoin.cz mining pool, Bitcoin Faucet, others... & Bitcoinica: 43554, Bitcoin.cz: 3094, Bitcoin Faucet: 4 & March 1st-2nd 2012\\\hline

Bitcoinica Hack & Bitcoinica & 18547 & May 12, 2012 \\\hline

Bitcoinica Theft & Bitcoinica & 40000 & July 13th, 2012 \\\hline 

CryptoRush Theft & CryptoRush & 950  & March 11th, 2014 \\\hline
\end{tabular}
\end{center}

The thefts above were selected out of a large collection since these thefts, in particular, include victims of entities collected in the entity tagging section of this project \ref{section-entity-tagging}.

\section{Existing Forensic Tools}\label{background-existing-tools}
There exist several digital forensic tools, both free and proprietary; we present in this section the information on these tools made available to the public (i.e. without making a purchase).  

\subsection{Blockchain Explorer}
Blockchain Explorer, accessible at https://www.blockchain.com/explorer, provides a wealth of information (available on the blockchain) about addresses, transactions and blocks. For instance, users can search by a public address in order to inspect the transactions associated with it. It also has features to provide charts on Bitcoin statistics, such as the network hash rate and exchange rates. However, in order to carry out a forensic investigation, it will often be necessary to follow the path of funds between addresses, which would be simplified if it were possible to do this in a graphical way; a graphical interface is not provided by Blockchain Explorer.

\begin{figure}[h!]
  \centering
  \includegraphics[width = 15cm]{./figures/chainanalysis}\\[0.5cm] 
  \caption{Blockchain Explorer's address investigation tool \protect \footnotemark}
\end{figure}
\footnotetext{https://www.blockchain.com/explorer}

\subsection{Chainanalysis}
Chainanalysis is a company which have built a proprietary software solution for digital blockchain forensics. We cannot know exactly the solutions Chainanalysis provide, or how, other than through the information used to advertise their product. 
\\\\
ChainAnalysis seem to have 3 main categories of customers and describes features they can offer to target each type of customer individually. The types of customers and their associated features are:
\begin{itemize}
    \item Financial Institutions: Focus on meeting Anti-money laundering (AML) and Know Your Customer compliance obligations. Tools for due-diligence and detecting suspicious activity. 
    \item Cryptocurrency Exchanges: Focus on AML obligations and due-diligence (similar to financial institutions offerings)
    \item Government: Features for suspect identification, criminal revenues and machine-learning based pattern recognition.
\end{itemize}

\subsection{Wallet Explorer}
Wallet Explorer is a website which provides similar capabilities to Blockchain Explorer in terms of inspecting addresses individually. However, Wallet Explorer additionally has a list of known entities and the public addresses they are known to be mapped to. This is extremely useful information in understanding patterns of use and the main players in the bitcoin network, however, this data is represented in quite a difficult to use format; it is not linked with network activity and therefore, on its own, it is not very useful in carrying out forensic investigations.

\begin{figure}[h!]
  \centering
  \includegraphics[width = 15cm]{./figures/walletexplorer}\\[0.5cm] 
  \caption{Wallet Explorers catalogue of wallet addresses \protect \footnotemark}
\end{figure}
\footnotetext{https://www.walletexplorer.com/}

\subsection{Blockpath} 
Blockpath is a website that claims to be a Bitcoin accounting tool. The feature of this site that was the most interesting is the graphical explorer. The graph allows you to explore the relationship between addresses. However, there does not appear to be any mapping of multiple public addresses to a single entity (i.e. clustering), which would be vital to gaining real insights when performing forensic analysis.  

\begin{figure}
  \centering
  \includegraphics[width = 15cm]{./figures/blockpath}\\[0.5cm] 
  \caption{Blockpath's graphical explorer displaying transactions between addresses \protect \footnotemark}
\end{figure}
\footnotetext{blockpath.com}

\subsection{Other Solutions}

There exist a number of companies claiming to offer similar solutions to Bitcoin analytics. 

\begin{itemize}
    \item \textbf{Elliptic} Similar to Chainanalysis, Elliptic offer proprietary software with main clients in finance and law enforcement. [See https://www.elliptic.co/what-we-do/bitcoin-forensics]. 
    \item \textbf{ScoreChain} Another proprietary software solution, claiming to perform Bitcoin analytics for compliance, forensics and CRM. [See https://bitcoin.scorechain.com]. 
    \item \textbf{Block Explorer:} Similar to Blockchain Explorer; providing visibility to information associated with single addresses and blocks. Also incorporates market information for many other cryptocurrencies and a cryptocurrency related news feature. [See https://blockexplorer.com/]. 

\end{itemize}


\section{Importing Blockchain Data}\label{design-db-previous-work}
There exists previous work in the domain of downloading the Bitcoin Blockchain. I first researched and assessed these implementations. 

\subsection{Bitcoin to Neo4J Tool: Open Source Project}
There exists an open source tool, built by the author of the website \url{learnmeabitcoin.com}, which populates a Neo4J database with the entire Bitcoin Blockchain \cite{RefWorks:doc:5c98e031e4b068320632cef2}. This tool requires a full Bitcoin node to be run in order to have the .dat files stored locally; the tool will parse the .dat files and write them, using Cypher queries, to the Neo4J database. This approach has the advantage that it is well respected in the community and has been cited a number of times by those seeking to achieve the same goal \cite{RefWorks:doc:5c98e0cde4b044512c0b8641}; however, the tool will take several weeks (apparently 60+ days) to complete the import. 

\subsection{Max Baylis : Imperial MSc Project 2018}\label{background-max-baylist-project}
Max undertook a project titled 'Blockchain Data Analytics and Health Monitoring' within the Department of Computing at Imperial \cite{RefWorks:doc:5c6bd151e4b041254f892045} which involved bulk extracting data from several blockchains and writing that data to a MongoDB database. He additionally used Kafka for streaming new blockchain updates and writing those updates to the database. This data was used for providing blockchain analytics data in a web-based environment \cite{RefWorks:doc:5c6bd151e4b041254f892045}. 

\subsection{TokenAnalyst : Medium Blog}
TokenAnalyst published a blog article on Medium with the title 'How to load Bitcoin to Neo4J in a day' \cite{RefWorks:doc:5c98e0cde4b044512c0b8641}. They describe their process of using RPC to fetch the data, writing the data to a 'data lake' in a compressed Arvo format, generating CSV's from the compressed files and feeding them to Neo4J's bulk import tool. Similar to Max's work, they then use Kafka for steaming the most recent Bitcoin data to a tool which writes it to Neo4J to keep the database up to date. 

\subsection{Blockchain2graph: Open Source Project}
A company Blockchain Inspector who are 'using Artificial Intelligence to fight fraud in the Blockchain' have open-sourced a tool they use with the claim that it extracts Bitcoin data and writes it to a Neo4J database \cite{RefWorks:doc:5cac6184e4b01c076c63e173}.

\subsection{Analysis of previous work}
TokenAnalyst's approach seems to best describe the ideal situation for this project; an efficient bulk import into Neo4J and a mechanism for keeping the database up to date. There exists no open source implementation for their work, only a high-level description of what they did in their blog post. Max's work, however, is available to me and his work in fetching data using RPC could prove useful to my project. The Bitcoin to Neo4J tool would not be a feasible approach and can be immediately ruled out due to the time requirement of several weeks for the tool to complete; this would not be acceptable for the timescale of this project. As for Blockchain Inspector's open-source solution, upon inspection of the code, it seems the bulk import process relies on an existing database containing the entire Bitcoin Blockchain and therefore seems to be more of a tool to migrate from a traditional database to a relational one (Neo4J). It is therefore not very applicable to my work, creating an intermediate database storing the Bitcoin Blockchain for the bulk import would be unnecessary and expensive.

\section{Know Your Customer}\label{background-kyc}
Know Your Customer (KYC) is a keystone in the fight against money laundering. It is the process of a business carrying out customer due-diligence measures and verifies the identity of its clients. This ensures the business is dealing with bonafide customers and organisations and helps identify suspicious behaviours or practices. These measures help satisfy anti-money laundering requirements (AML). 

\section{Privacy Enhanced Crypto-currencies}
\subsection{ZCash}\label{background-zcash}
ZCash has shielded transactions in which a zero-knowledge proof is used to verify the sender, recipient and amount of a transaction, without exposing the information on the public blockchain. Shielded transactions are encrypted on the blockchain, yet still verified by the network using zk-SNAKRK proofs. 

\subsection{MimbleWimble (Protocol)}\label{background-mimblewimble}
MimbleWimble is a blockchain protocol which addresses gaps in many blockchain protocols by using strong cryptographic primitives to provide good scalability, privacy and fungibility. Two current implementations of the MimbleWimble Protocol are Grin and Beam. Grin is developed in Rust and is a community-backed project. 

\subsection{Dash}
Dash is a cryptocurrency which offers a privateSend feature where users can mix the funds they are sending with others on the network; making it more difficult for a third party to identify where funds came from. There exist master nodes on the network that conduct coin mixing. 

\subsection{Monero}\label{background-monero}
Monero employs different privacy technologies to other crypto-currencies, such as Bitcoin and Ethereumm, which have transparent blockchains. Privacy-preserving technologies, such as \textit{ring signatures},\textit{ ring confidential transactions} and \textit{stealth addresses}, enable a verifiable blockchain without exposing details such as the sender, recipient or the amount of a transaction. These privacy features all exist by design for all transactions, such that all transactions made with Monero are made private through these features, rather than the selective privacy provided by some other privacy enhanced crypto-currencies such as Zcash. 

\subsubsection{Kovri}
Kovri is a private overlay network using routing and encryption allowing users to hide their geographical location and IP address. This avoids the need to use Tor which has semi-trusted authorities, who could have an overreaching influence of network consensus (and therefore in the determination of who can relay traffic). Kovri makes passive surveillance impossible. 

\subsubsection{Stealth Addresses}
Used to obscure the recipient address of a transaction. Stealth addresses are one-time (ephemeral) public keys. Observers cannot infer the recipient from the stealth address, however, the sender can verify that the payment was sent by then. A Monero wallet has a public view key and a public send key. When Alice sends Monero to Bob, Alice will use Bob's public view and send key, in addition to some random data in order to generate the one-time ephemeral key (stealth address). The entire blockchain can see the stealth address. Bob, with his wallet's private view key, can identify the output of the transaction sent to him on the blockchain. Bob will be able to generate a one-time private key that corresponds to the one-time public key in order to spend the funds. 

\subsubsection{Ring Signatures}
Used to obscure the sender. A group of possible signers are fused together to produce a possible signature. The actual signer is a one time spend key that corresponds with an output being spent from the senders' wallet; the non-signers in the ring are past transaction outputs pulled from the blockchain being used as decoys. These outputs combined make up the inputs of a transaction in which it is indistinguishable between the actual signer and the decoy signers. Key imaged are used to detect double-spends. 

\subsubsection{Ring Confidential Transaction (CT) Signatures}
Obscures the amount of Monero sent in a transaction. 
Before Ring CT's, Monero would require transaction amounts to be split up into smaller denominations and split amongst ring signatures to ensure there would be enough ring signers (since all signers of a ring must have outputs of the same value). This means an observer can see the amounts sent in transactions. Ring CT transactions obscure the value of outputs, but now the sender must commit to the value of an output (Pedersen commitment scheme) but without publicly disclosing the amount being sent. 



