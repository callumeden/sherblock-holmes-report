\section{Web UI Investigation Tool}

\subsection{Technology}
\subsubsection{Angular 6}

\subsubsection{D3}

\subsection{Search By Address}
An investigation can be initiated by searching for a particular bitcoin address. The search form as shown in figure \ref{fig:neo4j-screenshot-search-form}, when submitted, loads the investigation view displaying the address and it's immediate neighbours as nodes in the graph, also detailing the relationship types each link represents, as shown in figure \ref{fig:neo4j-screenshot-basic-address-nodes}. 

\begin{figure}[h!]
  \centering
  \includegraphics[width = 15cm]{./figures/ui-screenshots/search-address-form}\\[0.5cm]
  \caption{A screenshot of the address search form feature.}
  \label{fig:neo4j-screenshot-search-form}
\end{figure}

\begin{figure}[h!]
  \centering
  \includegraphics[width = 15cm]{./figures/ui-screenshots/graph-basic}\\[0.5cm] 
  \caption{A screenshot of the search result produced by the search shown in figure \ref{fig:neo4j-screenshot-search-form}}
  \label{fig:neo4j-screenshot-basic-address-nodes}
\end{figure}

\subsection{Node information on hover}
Information associated with each type of node is shown on hover. For example, hovering over a block will provide the time it was mined, its hash and the historical exchange rates at the time of mining (see fig \ref{fig:block-info-on-hover}). The node being hovered over will expand in size to indicate which node the information is being displayed for. There is also an option to dismiss the information box using the cross symbol in the top right corner. 
\\\\
Additionally, clicking on each field in the node information box automatically copys that data to the users clipboard. This further improves the user experience as there exists less steps to copying data (such as long ID's or hashes) for further investigation. The data copied will also be copied without the extraneous surrounding data such as labels, currency symbols or data formatting (specifically, selecting dates will copy the epoch time in milliseconds to the clipboard). 

\begin{figure}[h!]
  \centering
  \includegraphics[width = 15cm]{./figures/ui-screenshots/block-info}\\[0.5cm] 
  \caption{A screenshot of block information being displayed when hovering over a block node}
  \label{fig:block-info-on-hover}
\end{figure}

\subsection{Link Data}
A primary use of the investigation tool will be to investigate the flow of funds; it is therefore important to make this information visible and digestible. Therefore, each link representing a flow of funds (between \texttt{OUTPUT} and \texttt{TRANSACTION} nodes) displays the value of the funds in BTC and whichever fiat currency the user selects as their preference when searching (GBP, EUR, USD). The link data also provides a timestamp representing the time the transaction was included in the blockchain; this provides context to the value of the output respective to their fiat currencies. 
\\\\
The screenshot shown in figure \ref{fig:link-data} exemplifies the importance of a fiat currency conversion: the same unit of 0.5 BTC is output by transaction \texttt{432f18...} as is input into transaction \texttt{96455e...}, however their equivalent fiat value in USD is dramatically different at $\$1,688$ when the output is produced and $\$2,102$ when the output is spent. The timestamp will also provide context around the timeline of this transfer of funds, in addition to possibly explaining differences in fiat currencie exchange rates.

\begin{figure}[h!]
  \centering
  \includegraphics[width = 15cm]{./figures/ui-screenshots/link-data}\\[0.5cm] 
  \caption{A screenshot link data between two transaction nodes (green) and an output node. 
  Transactions IDs are: \texttt{432f18aa46626934c45805046f9c9791fb60bd41da1eb951b47f73bb3b8c7484},\\\texttt{96455e8b6a877f273d14aa13ecb1971577c0d17716b7028c9809c7ca3e1f6e3c}}
  \label{fig:link-data}
\end{figure}

\subsection{Traverse the Graph}
Double clicking on a node initiates a request to load that nodes immediate neighbours. Once the request receives a response, the nodes neighbours are added to the existing graph as new nodes and the necessary links are created. Each new node can then also be double clicked and the process will recurse. 
\\\\
Sometimes the time between initiating a request and receiving a response isn't near instantaneous, perhaps due to demand on the database or a greater response size. Therefore, to feedback to the user that a request has been initiated, to prevent them trying to issue multiple requests or thinking that a node has no neighbours, a node will pulse in size while a request is pending. 
\\\\
Additionally, in an investigation view with many nodes, it may not be clear which nodes already have had their neighbouring nodes requested (those without neighbours, for instance). Therefore, nodes will have a thick black border until their nodes have been expanded (see both green transaction nodes in fig \ref{fig:link-data} and will have their border removed once expanded (see the orange output node in fig \ref{fig:link-data}).

\subsection{Link Dependant Colour and Size of nodes}
To indicate the differences in a nodes connectivity among the many other nodes in the graph, a nodes colour and size adjusts based on the number of links it has outgoing and incident on it. The rate at which a nodes colour and size changes is normalised by the total number of links in the graph. A node with 30\% of the links in a graph of 10 links should have the same size/colour as a node with 30\% of the links in a graph with 100 links. This feature helps with visualising potentially more important nodes in a graph, and also assists with visual arrangement, by creating a greater circumference to place neighbouring nodes around a node with many links.



\subsection{Add nodes with complementary}

\subsection{Filter Date/Time}

\subsection{Filter by value, several currencies}

\subsection{Enable Multi-input clustering view}

\subsection{Configure how many nodes to display}

\subsection{User Input Validation}

\subsection{Loading feedback states}

\subsection{Limiting nodes}

